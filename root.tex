%%%%%%%%%%%%%%%%%%%%%%%%%%%%%%%%%%%%%%%%%%%%%%%%%%%%%%%%%%%%%%%%%%%%%%%%%%%%%%%%
%2345678901234567890123456789012345678901234567890123456789012345678901234567890
%        1         2         3         4         5         6         7         8

\documentclass[letterpaper, 10 pt, conference]{bib/ieeeconf}
%\documentclass[a4paper, 10pt, conference]{ieeeconf}      % Use this line for a4 paper
\IEEEoverridecommandlockouts                              % This command is only needed if
                                                          % you want to use the \thanks command
\overrideIEEEmargins                                      % Needed to meet printer requirements.


%XXX: Packages
\usepackage[normalem]{ulem}	                        % underlining!
\usepackage[table,usenames,dvipsnames]{xcolor}      % color
\usepackage{extarrows}                              % http://ctan.org/pkg
\let\labelindent\relax
\usepackage{enumitem}                               % [inline]
% Math
\let\proof\relax
\let\endproof\relax
\usepackage{amsmath,amsthm,amssymb,amsfonts,dsfont} % math
\usepackage{algorithm,algorithmicx,listings}        % algorithms
\usepackage[noend]{algpseudocode}			        % necessary for algorithmicx
% Figures
\usepackage{graphicx}
\usepackage{subfigure}
\usepackage{tabularx}
\usepackage{amsmath}
\usepackage{multirow,multicol, array}
\usepackage[font={small}]{caption}   %onehalfspacing
\captionsetup[algorithm]{font=small}
%\setlength{\belowcaptionskip}{-3.5pt}
%\setlength{\abovecaptionskip}{2pt}
%\usepackage{subcaption}
%\usepackage[font={small}]{subcaption}
\let\appendices\relax
\usepackage[titletoc,title]{appendix}
\usepackage[breaklinks=true, colorlinks, bookmarks=true, citecolor=Black, urlcolor=Violet,linkcolor=Black]{hyperref}

% XXX: COMMANDS:
\def\argmin{\mathop{\arg\min}\limits}	%
\def\argmax{\mathop{\arg\max}\limits}	%
\newcommand{\longeq}[2]{\xlongequal[\!#2\!]{\!#1\!}}
% #1 = top; #2 = bottom; #3 = inequality (<,>,\leq,\geq)
\newcommand{\longineq}[3]{\overset{#1}{\underset{#2}{#3}}}
\newcommand{\indicator}{\mathds{1}}
\newcommand{\ceil}[1]{\lceil#1\rceil}
\newcommand{\floor}[1]{\lfloor#1\rfloor}
\DeclareMathOperator{\tr}{tr}
\newcommand{\TODO}[1]{{\color{red}#1}}
\newcommand{\prl}[1]{\mathopen{}\left(#1\right)\mathclose{}}
\newcommand{\brl}[1]{\mathopen{}\left[#1\right]\mathclose{}}
\newcommand{\crl}[1]{\mathopen{}\left\{#1\right\}\mathclose{}}
\newcommand{\scaleMathLine}[2][1]{\resizebox{#1\linewidth}{!}{$\displaystyle{#2}$}}
\newcommand{\T}{\mathsf{T}}

% XXX: ENVIRONMENTS:
\newtheorem{theorem}{Theorem}
\newtheorem{proposition}{Proposition}
\newtheorem{corollary}{Corollary}
\newtheorem{definition}{Definition}
\newtheorem{assumption}{Assumption}
\newtheorem*{assumption*}{Assumption}
\newtheorem{remark}{Remark}
\newtheorem*{remark*}{Remark}
\newtheorem*{problem*}{Problem}
\newtheorem{problem}{Problem}
\newtheorem{lemma}{Lemma}
\addtolength{\textfloatsep}{-3mm}
\addtolength{\intextsep}{-2mm}	   % reduce space between intext fig & text
%\addtolength{\floatsep}{-1mm}		 % reduce the space between figures

\title{\LARGE \bf
Safe Guaranteed Motion Planning for Quadrotors\\ with Simultaneously Modeling and Exploring
}
\author{Sikang Liu, Nikolay Atanasov, Kartik Mohta, and Vijay Kumar% <-this % stops a space
\thanks{This work is supported in part by ARL \# W911NF-08-2-0004, DARPA \# HR001151626/HR0011516850, ARO \# W911NF-13-1-0350, and ONR \# N00014-07-1-0829. The authors are with the GRASP Laboratory, University of Pennsylvania. Email: \texttt{\{sikang, atanasov, kmohta, kumar\}@seas.upenn.edu}}
}
% $^{1}$
% \thanks{$^{1}$S. Liu, N. Atanasov, K. Mohta, and V. Kumar are with the GRASP Laboratory, University of Pennsylvania, USA }


%%%%%%%%%%%%%%%%%%%%%%%%%%%%%%%%%%%%%%%%%%%%%%%%%%%%%%%%%%%%%%%%%%%%%%%%%%%%%%%%
\begin{document}
\maketitle
\begin{abstract}
In traditional search-based planning approach, 

\end{abstract}


%\input{tex/Intro}
%\input{tex/Problem}
%\input{tex/LQMT}
%\input{tex/DeterministicShortestPath}
%\input{tex/Solution}

%\input{tex/Trajectory}
%\input{tex/Exp}
%\input{tex/Conclusion}
%\appendices
%\input{tex/Appendix}
%\input{tex/HeuristicNikolay}

%%%%%%%%%%%%%%%%%%%%%%%%%%%%%%%%%%%%%%%%%%%%%%%%%%%%%%%%%%%%%%%%%%%%%%%%%%%%%%%%
%\bibliographystyle{bib/IEEEtran}
%\bibliography{bib/references}


%\begin{thebibliography}{99}
%\bibitem{mellinger2011} Mellinger, Daniel, and Vijay Kumar. "Minimum snap trajectory generation and control for quadrotors." Robotics and Automation (ICRA), 2011 IEEE International Conference on. IEEE, 2011.
%\bibitem{mueller2015} Mueller, Mark W., Markus Hehn, and Raffaello D'Andrea. "A computationally efficient motion primitive for quadrocopter trajectory generation." IEEE Transactions on Robotics 31.6 (2015): 1294-1310.
%\bibitem{bouktir2008} Bouktir, Y., M. Haddad, and T. Chettibi. "Trajectory planning for a quadrotor helicopter." Control and Automation, 2008 16th Mediterranean Conference on. Ieee, 2008.
%\bibitem{Jamieson2016} J. Jamieson and J. Biggs, "Near minimum-time trajectories for quadrotor UAVs in complex environments," 2016 IEEE/RSJ International Conference on Intelligent Robots and Systems (IROS), Daejeon, 2016, pp. 1550-1555.
%doi: 10.1109/IROS.2016.7759251
%\bibitem{richter2013} Richter, Charles, Adam Bry, and Nicholas Roy. "Polynomial trajectory planning for aggressive quadrotor flight in dense indoor environments." Robotics Research. Springer International Publishing, 2016. 649-666.
%\bibitem{sikang2016} Liu, Sikang, et al. "High speed navigation for quadrotors with limited onboard sensing." Robotics and Automation (ICRA), 2016 IEEE International Conference on. IEEE, 2016.
%\bibitem{chen2016} Chen, Jing, Tianbo Liu, and Shaojie Shen. "Online generation of collision-free trajectories for quadrotor flight in unknown cluttered environments." Robotics and Automation (ICRA), 2016 IEEE International Conference on. IEEE, 2016.
%\end{thebibliography}

\end{document}


